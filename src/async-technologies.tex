\documentclass[11pt, a4paper]{scrartcl}
%---------------------------------------------------------
% Package imports
%---------------------------------------------------------

% Language and encoding
\usepackage[utf8]{inputenc}
\usepackage{csquotes}
\usepackage[USenglish]{babel}
\usepackage[T1]{fontenc}

% Hyperrefs / links
\usepackage{hyperref}

% Tables
\usepackage{tabulary}
\usepackage{tabularx}
\usepackage{multirow}
\renewcommand{\arraystretch}{1.4} % increase spacing for table rows
\usepackage{longtable}
\usepackage{booktabs}

% Graphics / figures
\usepackage{graphicx}
\usepackage{wrapfig}

% Used to restrict floating of tables/illustrations
\usepackage{float}
\usepackage{placeins}

% Graphics / figures
\usepackage{graphicx}
\usepackage{wrapfig}
\usepackage{tikz}

% CSV Support
\usepackage{csvsimple}

% Footnotes + captions
\usepackage{caption}
\usepackage{footnote}

% Bibliography - Biblatex
\usepackage[style=dhbw_ibim,firstinits=true,backend=biber]{biblatex}
\bibliography{bibliography} % Bibliography file

% Abbreviations
\usepackage{nomencl}
\setlength{\nomlabelwidth}{.25\hsize}
\renewcommand{\nomlabel}[1]{#1 \dotfill}
\makenomenclature

% The Euro Symbol Package for LaTeX
\usepackage[right]{eurosym}

%\usepackage[usenames,dvipsnames]{xcolor} 
% color definitions for listings
\definecolor{lstkeyword}{RGB}{0,0,110}
\definecolor{lstidentifier}{RGB}{0,0,0}
\definecolor{lstcomment}{RGB}{70,70,70}
\definecolor{lststring}{RGB}{34,84,34}

% Listings
\usepackage{listings}

\lstdefinelanguage{JavaScript}{
  keywords={typeof, new, true, false, catch, function, return, null, catch, switch, var, if, in, while, do, else, case, break},
  keywordstyle=\color{blue}\bfseries,
  ndkeywords={class, export, boolean, throw, implements, import, this},
  ndkeywordstyle=\color{darkgray}\bfseries,
  identifierstyle=\color{black},
  sensitive=false,
  comment=[l]{//},
  morecomment=[s]{/*}{*/},
  commentstyle=\color{purple}\ttfamily,
  stringstyle=\color{red}\ttfamily,
  morestring=[b]',
  morestring=[b]"
}

\lstset{ %
	basicstyle=\footnotesize,       % the size of the fonts that are used for the code
	numbers=left,                   % where to put the line-numbers
	numberstyle=\footnotesize,      % the size of the fonts that are used for the line-numbers
	lineskip={0.8pt},				% line spacing
	stepnumber=1,                   % the step between two line-numbers. If it's 1
	numbersep=10pt,                 % how far the line-numbers are from the code
	backgroundcolor=\color{white},  % choose the background color. You must add
	showspaces=false,               % show spaces adding particular underscores
	showstringspaces=false,         % underline spaces within strings
	showtabs=false,                 % show tabs within strings adding particular underscores
	frame=single,                   % adds a frame around the code
	tabsize=2,                      % sets default tabsize to 2 spaces
	captionpos=b,                   % sets the caption-position to bottom
	breaklines=true,                % sets automatic line breaking
	breakatwhitespace=false,        % sets if automatic breaks should only happen at whitespace
	title=\lstname,                 % show the filename of files included with \lstinputlisting;
	keywordstyle=\color{lstkeyword}\bfseries,
	identifierstyle=\color{lstidentifier},	
	commentstyle=\color{lstcomment}, 	
	stringstyle=\color{lststring}\ttfamily,		% typewriter type for strings	
}

%--------------------------------------------------------------------
% Definitions for the \listofappendices command
%--------------------------------------------------------------------
\usepackage[title]{appendix}

% Use a counter for appendices
\newcounter{appendixctr} 

% Define commands to create a separate list of appendices
\makeatletter 

% Create an auxiliary file for the appendix list with an .app extension
\newcommand{\listofappendices}{\@starttoc{app}}

% Define command to be used to insert an appendix string into the app
% table with entry type myappendix
\newcommand{\addtoappendixlist}[1]{\addcontentsline{app}{myappendix}{#1}}

% Define the command actually used to start an Appendix - generate a
% header and then insert the string name into the table.
\newcommand{\appsection}[1]{%
	%\addtocounter{appendixctr}{1}
	\refstepcounter{appendixctr}
	\subsection*{Appendix \arabic{appendixctr}:\hspace{15pt}#1}
	\addtoappendixlist{Appendix \arabic{appendixctr}:\hspace{10pt}#1}
}

% Define the required command needed to format a myappendix type entry in
% the table.  It uses the standard \@dottedtocline command to actually format
% the entry in the table.
\newcommand{\l@myappendix}[2]{\@dottedtocline{1}{1.5em}{2.3em}{#1}{#2}}
\makeatother


%--------------------------------------------------------------------
% Standard Names / Translations
%--------------------------------------------------------------------
% Table of Contents
\renewcommand\contentsname{Table of Contents}
% List of Abbreviations
\renewcommand{\nomname}{List of Abbreviations}
% Abstract
\renewcommand{\abstractname}{Abstract}
% Appendix
\renewcommand{\appendixname}{Appendix}
% renewal of appendixpagename needed to set proper font size of title
\renewcommand*\appendixpagename{\Large Appendices }
% ToC name for appendices
\renewcommand{\appendixtocname}{Appendices}
% List of Appendices
\newcommand{\listappendixname}{List of Appendices}
% Figure (caption)
\renewcommand{\figurename}{Figure}
% List of Figures
\renewcommand{\listfigurename}{List of Figures}
% List of Tables
\renewcommand{\listtablename}{List of Tables}
% References (for articles)
\renewcommand{\refname}{References}
% Lists of References
\newcommand{\listrefname}{Lists of References}
% Table (captions)
\renewcommand{\tablename}{Table}
% Listings (caption)
\renewcommand\lstlistingname{Listing}
%  List of Listings
\renewcommand\lstlistlistingname{List of Listings}
% Autoref name for listings
\def\lstlistingautorefname{Lst.}

%---------------------------------------------------------
% Settings & Layout
%---------------------------------------------------------

% Border definitions
\usepackage[paper=a4paper,
			left=35mm,
			right=15mm,
			top=25mm,
			bottom=20mm]{geometry}


% Substitutions for re-use in several places
\newcommand{\papertitle}{Evaluation of Asynchronous Server Technologies}
\newcommand{\papertype}{Seminar Paper}
\newcommand{\paperauthor}{Rocco Schulz, Robert Wawrzyniak, Joe Boden, Max V��kler, Can Paul Bineytioglu}
% Paper
\newcommand{\pdfkeywords}{node.js vert.x asynchronous network programming} % separated by whitespace

\newcommand{\studyprogram}{Int. Business Information Management}
\newcommand{\studycourse}{WWI 2010 I}
\newcommand{\schoolof}{Business}


% Information for distiller pdftex:
\hypersetup{ 
	pdfauthor={Rocco Schulz},
	pdftitle={\papertitle},
	pdfsubject={\papertype - Program:  \studyprogram| Course: \studycourse},
	pdfkeywords = \pdfkeywords,
	linkcolor={black},
	pdfview=FitV,     % FitH
	pdfstartview=FitV,
	urlcolor=black,   
	breaklinks=true,   
	citebordercolor=0 0 0, 
	filebordercolor=0 0 0,
	linkbordercolor=0 0 0,
	menubordercolor=0 0 0,
	urlbordercolor=0 0 0,
	pdfhighlight=/I,
	pdfborder=0 0 0,   % no boxes for hyperrefs
} 


% Depth for the table of contents
\setcounter{tocdepth}{3}
\setcounter{secnumdepth}{3}

% No clubpenalty & widowpenalty
\clubpenalty = 10000
\widowpenalty = 10000 
\displaywidowpenalty = 10000


%---------------------------------------------------------
% Contents
%---------------------------------------------------------



\title{\papertitle}
\author{Rocco Schulz, Max Vökler, Joe Boden,\\ Robert Wawrzyniak, Can Paul Bineytioglu\\\\
	Corporate State University\\Baden-Wuerttemberg - Stuttgart}

\date{\today}

\begin{document}

\maketitle


\begin{abstract}
This paper evaluates asynchronous server technologies. Strenghts and weaknesses
of asynchronous programming models are elaborated and a proof of concept 
based on node.js and vert.x is used to evaluate non-functional attributes such as
maintainability. \ldots
\end{abstract}

\section{Introduction}
In traditional Web applications a connection's lifespan starts with a request and
ends with the servers response. Each connection is assigned a thread which
processes the request and terminates afterwards.
However in order to allow real-time\footnote{proper real time, as opposed to
ancient definitions. TODO: reference} communication between client and server it
is necessary to extend the connection's lifespan to the duration of the whole
session which results in many simultaniously open connections.
Assigning one thread for each open connection would cause a large overhead for
the server even though most threads would be inactive. This is due to the data
associated with each thread and the process of context switching between all
threads.\\
Relatively young frameworks such as Node.js and Vert.x try to address this issue 
by providing a completely asynchronous programming model which allows associating
multiple simultanious connections with a single thread by using an event-driven approach.

%Goal of this paper and structure
This paper elaborates the concepts behind these young frameworks and analyses their 
technical strengths and weaknesses. Furthermore non-functional attributes will be
evaluated based on two sample implementations in Node.js  and Vert.x.

\section{Setting the context}
\label{setting_the_context}

\subsection{Comparison between Asynchronous and Synchronous Processing}
\label{comparison}

%Synchronous calls -> idling threads -> 
A common case in programming is access to I/O.
In synchronous processing a running thread needs to wait for the completion of
the I/O operation before it can continue.
The thread is in an idle state while it is waiting which allows another process 
or thread to occupy the CPU in the meanwhile.\\

% writing multithreaded code is not trivial
In multithreaded applications several threads can run simultaniously within one 
process. Several threads can access shared memory concurrently which can
cause inconcistent states. This can be avoided by synchronizing threads - e.g.
with locks. This means that programmers need to take into account every possible
execution order to effectively avoid program defects such as data races and 
deadlocks.\footcite[Cf.][10]{Breshears_2009}
This can be time consuming and potentially results in error-prone code.

A typical synchronous call is provided in listing \ref{lst:synchronous_call}. The
contents of a file are read and displayed afterwards. The programm is blocked until the
read operation has finished.

% reading a file and displaying it, pseudo code
\lstinputlisting[caption={Pseudocode: Synchronously reading and displaying a file's contents},
label=lst:synchronous_call]{lst/synchronous_call.txt}


Asynchronous programming style uses a different concept. The flow of an
application is determined by events, which is why this style is also called
event-driven programming.\footcite[Cf.][16]{teixeira_2012} In listing \ref{lst:asynchronous_call} the call
to the \textit{readAsText} function is done asynchronously. There is no
return value - instead an event handler is provided as a second argument.
This function is also referred to as a callback function. It is called
as soon as the read operation has completed.

% same example as above but with callbacks / events
\lstinputlisting[caption={Pseudocode: Asynchronously reading and displaying a file's contents},
label=lst:asynchronous_call]{lst/asynchronous_call.txt}

This concept is coupled with an event loop, which is a single thread that is
running inside the main process.
The loop constantly checks for new events. When an event is detected, the loop
invokes the corresponding callback function. The callback is processed in the
same thread which means that there is at most one callback running at a time.
The event loop continues when the callback has completed. As a result the
developer does not need to take care of concurrency issues during development.
But the developer's task is to write light event handlers that can be processed
quickly as every callback is an interruption of the event processing in the
event loop. \footcite[Cf.][]{Croucher_2010} Memory or processor intense callbacks
can lead to growing queues of unserved events which eventuelly results
in a slow application or service\footcite[Cf.][48]{teixeira_2012}.


% implication: 
% all code should be as non-blocking and asynchronous as possible
% using blocking APIs inside the asynchronous code can cause blocking of the event loop
% which is evil. This is why node.js is based on javacript. Other languages already have
% lots of modules with blocking APIs which could confuse developers.



\subsection{Existing Asynchronous Frameworks}
\label{existing_frameworks}


\subsubsection{Overview}
\label{frameworks_overview}

\begin{savenotes} %alloows proper citation marks inside tables
\begin{table}[h]
\centering
\begin{tabular*}{\textwidth}{p{0.2\textwidth} p{0.15\textwidth} p{0.57\textwidth}}
\toprule
\textbf{Name} & \textbf{Language/s} & \textbf{Description} \\
\midrule 
Twisted			& Python			& ``Twisted is an event-driven networking engine written in 
									  Python and licensed under the open source MIT license"\footcite[Cf.][]{Twisted_2012}.
									  Twisted is not fully threadsafe and implements the event
									  loop via a library.\\
									  
EventMachine	& Ruby				& Similar to Twisted.\\

Node.js			& JavaScript		& Node is based on Chrome's JavaScript runtime \textit{V8}.
									  The platform is fully event-driven and offers core
									  functionalities for network programming. Its functionality
									  can be extended with a large number of modules using an
									  integrated package management system.
									  Node.js started development in 2009 and the community
									  is growing since that.\footcite[Cf.][]{Mashtable_2011}\\
									  
Vert.x			& JavaScript, Java, Python, Groovy, Ruby, Coffeescript		
									& A JVM based platform for network applications which is
									  inspired by Node.js. Vert.x comes with its own event
									  bus system that allows distributing applications
									  among multiple network nodes. Support for the languages
									  Scala and Clojure is scheduled for future releases.\footcite[Cf.][]{vertx_2012}\\
\bottomrule 
\end{tabular*}
  \captionof{table}{Existing asynchronous programming frameworks}
  \label{tab:existing_frameworks}
\end{table}
\end{savenotes}



\subsubsection{Node.js}
\label{node.js}

%


\subsubsection{Vert.x}
\label{vert.x}

Vertx is a polyglot that runs on the JVM (Java Virtual Machine). It is hence
possible to scale over available cores without manually forking multiple
servers.\\ %TODO: how are requests distributed in the background?
The application API is exposed in multiple programming lanugages (see table
\ref{tab:existing_frameworks}).
% Implication: use case, code reuse from different legacy systems written in
% different languages

In Vert.x the smallest available deployment unit is a verticle, which runs
inside an instance. Each instance runs singlethreaded.
Multiple verticles can be run inside one instance as depicted in figure
\ref{fig:vertx_constructs}.
%http://vertx.io/core_manual_java.html scaling tcp servers

\begin{figure}[h]
	\centering
	\setlength\fboxsep{2pt}
	\fbox{
	\includegraphics[width=0.4\textwidth]{img/vertx_constructs.pdf}
	}
	\caption{Abstracted deployment units of Vert.x}
	\label{fig:vertx_constructs}
\end{figure}

When multiple instances are run on one machine, Vert.x automatically distributes incoming
requests among all running instances in a round-robin fashion, so that each
vert.x verticle instance remains single threaded.

Vert.x also includes a distributed event bus, which enables verticles to
communicate with each other, either within the same instance or across different
instances. The event bus allows direct communication with in-browser JavaScript as well.\\
% Implication: ideal for real time applications. easy communication between client and server using JS
Vert.x allows to run IO heavy tasks in separate worker threads that reside in a
so-called background pool as these tasks would otherwise block the event loop as
described in section \ref{comparison}.

The core functionality of Vert.x covers basic networking tasks and protocols,
web servers, clients, access to the file system, shared maps and the event bus.
The core libraries of Vert.x can be embedded in any JVM program for reuse in
larger projects.\\
As opposed to Node.js, the core functionality and API can be considered quite
static as changes need to be done in all supported
languages.\footcite[Cf.][]{vertx_2012}\\ % The core can be extended with
Additional features that are provided by optional modules that can be obtained
over a public git-based module repository.\footcite[Cf.][]{vertx_mod_2012}.
The repository currently contains 16 distinct modules in different
versions.\footcite[Cf.][]{Vertx_repository_2012}

An extensive online documentation is available for all supported languages.
Additionally code examples are available for all languages and most features in
a public repository\footcite[Cf.][]{Fox_2013}.\\
Vert.x is open source and licensed under the Apache Software License
2.0\footcite{See \url{http://www.apache.org/licenses/LICENSE-2.0.html}}, so that
commecial redistribution in closed source projects should not be an issue.





\section{Areas of Application}
\label{areas_of_application}

The non-blocking nature of asynchronous calls is important in all types of
applications that need to handle a large number of requests in real time.

Some success stories are available on \url{http://nodejs.org/}.


\section{Exemplary Implementations}
\label{exemplary_implementations}

A simple web form application has been implemented in Node.js and Vert.x to
further analyze non-functional requirements and collect practical experience
with these frameworks.

\subsection{Software Description}
\label{software_description}
Brief description of the insurance fee calculator

\subsection{Software Design}
\label{software_design}
High level design\\
Interface description, and differences between Node.js and Vert.x

\subsection{Software Implementation}
\label{software_implementation}
Complications or any other notes on the implementation process that might be of
importance for the evaluation.





\section{Evaluation of Non-functional Attributes}
\label{evaluation_nonfunctional}

\subsection{Maintainability}
\label{maintainability}
Language: JavaScript is wide spread, same for Java. However JavaScript offers
better flexibility and native constructs for these types of applications.
(Java 8 might improve things a bit).
Node's API is undergoing backwards incompatible changes from time to time.
It is desired to updated some components of a Node based application from
time to time.

\subsection{Integration}
\label{integration}
Vert.x and Node.js are supposed to be used as fully event driven standalone
applications that can be extended with event-driven modules.
However it might be desired to reuse existing software that is not fully event
driven.
Dont make use of blocking apis.
Consider connecting that software with a message bus.
Separate IO intense tasks into ``web workers'' or similar constructs.

\subsection{Scalability}
\label{scalability}
Support for single machine scaling using multiple threads exists.
Deployment on distributed systems differs in node.js and vert.x.
Vert.x. uses its own communication bus to share information between verticles.


\section{Conclusion}
\label{conclusion}


%---------------------------------------------------------
% BIBLIOGRAPHY
%---------------------------------------------------------

\newpage

\addsec{\listrefname} %addsec: komaskript befehl - keine Nummerierung
% Set headings for bibliographies
\defbibheading{literatur}{\subsection*{List of Literature}}
\defbibheading{www}{\subsection*{List of Internet and Intranet Resources}}

% Show bibliographies
\printbibliography[type=book,heading=literatur]
% warning: includes ALL types but online.
% Check your bibfile to avoid wrong entries here
\printbibliography[nottype=book,heading=www]

\end{document}

