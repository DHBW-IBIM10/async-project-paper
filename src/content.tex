

\title{\papertitle}
\author{Rocco Schulz, Max Vökler, Joe Boden,\\ Robert Wawrzyniak, Can Paul Bineytioglu\\\\
	Corporate State University\\Baden-Wuerttemberg - Stuttgart}

\date{\today}

\begin{document}

\maketitle


\begin{abstract}
This paper evaluates asynchronous server technologies. Strenghts and weaknesses
of asynchronous programming models are alaborated and a proof of concept 
based on node.js and vert.x is used to evaluate non-functional attributes such as
maintainability. \ldots
\end{abstract}

\section{Introduction}
Lorem ipsum dolor sit amet.\footcite[Cf.][30]{fettig_2005}

\paragraph{Key Terms}
In general asynchronous calls can be defined as non-blocking function or method calls that allow the caller to continue while the callee handles the task.
The retrieval of results is usually done by callbacks that are triggered by certain events (such as the completion of the task). This callback approach is referred to as event-driven.

TODO: Benefits of this approach vs synchronous (literature: distributed systems)

TODO: list of recent asynchrounous programming models and frameworks such as vert.x, node.js, twisted, EventMachine, etc.

TODO: purpose of this paper / goal definitions
This paper focusses on the more recent representatives of asynchronous programming models, in particular node.js and vert.x. 


\section{Areas of Application} \label{areas_of_application}




\section{Exemplary Implementations} \label{exemplary_implementations}

lalalala

