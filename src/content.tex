

\title{\papertitle}
\author{Rocco Schulz, Max Vökler, Joe Boden,\\ Robert Wawrzyniak, Can Paul Bineytioglu\\\\
	Corporate State University\\Baden-Wuerttemberg - Stuttgart}

\date{\today}

\begin{document}

\maketitle


\begin{abstract}
This paper evaluates asynchronous server technologies. Strenghts and weaknesses
of asynchronous programming models are elaborated and a proof of concept 
based on node.js and vert.x is used to evaluate non-functional attributes such as
maintainability. \ldots
\end{abstract}

\section{Introduction}
In traditional Web applications a connection's lifespan starts with a request and
ends with the servers response. Each connection is assigned a thread which
processes the request and terminates afterwards.
However in order to allow real-time\footnote{proper real time, as opposed to
ancient definitions. TODO: reference} communication between client and server it
is necessary to extend the connection's lifespan to the duration of the whole
session which results in many simultaniously open connections.
Assigning one thread for each open connection would cause a large overhead for
the server even though most threads would be inactive. This is due to the data
associated with each thread and the process of context switching between all
threads.\\
Relatively young frameworks such as Node.js and Vert.x try to address this issue 
by providing a completely asynchronous programming model which allows associating
multiple simultanious connections with a single thread by using an event-driven approach.

%Goal of this paper and structure
This paper elaborates the concepts behind these young frameworks and analyses their 
technical strengths and weaknesses. Furthermore non-functional attributes will be
evaluated based on two sample implementations in Node.js  and Vert.x.

\section{Setting the context}
\label{setting_the_context}

\subsection{Comparison between Asynchronous and Synchronous Processing}
\label{comparison}

%Synchronous calls -> idling threads -> 
A common case in programming is access to I/O.
In synchronous processing a running thread needs to wait for the completion of
the I/O operation before it can continue.
The thread is in an idle state while it is waiting which allows another process 
or thread to occupy the CPU in the meanwhile.\\

% writing multithreaded code is not trivial
In multithreaded applications several threads can run simultaniously within one 
process. Several threads can access shared memory concurrently which can
cause inconcistent states. This can be avoided by synchronizing threads - e.g.
with locks. This means that programmers need to take into account every possible
execution order to effectively avoid program defects such as data races and 
deadlocks.\footcite[Cf.][10]{Breshears_2009}
This can be time consuming and potentially results in error-prone code.

A typical synchronous call is provided in listing \ref{lst:synchronous_call}. The
contents of a file are read and displayed afterwards. The programm is blocked until the
read operation has finished.

% reading a file and displaying it, pseudo code
\lstinputlisting[caption={Pseudocode: Synchronously reading and displaying a file's contents},
label=lst:synchronous_call]{lst/synchronous_call.txt}


Asynchronous programming style uses a different concept. The flow of an
application is determined by events, which is why this style is also called
event-driven programming.\footcite[Cf.][16]{teixeira_2012} In listing \ref{lst:asynchronous_call} the call
to the \textit{readAsText} function is done asynchronously. There is no
return value - instead an event handler is provided as a second argument.
This function is also referred to as a callback function. It is called
as soon as the read operation has completed.

% same example as above but with callbacks / events
\lstinputlisting[caption={Pseudocode: Asynchronously reading and displaying a file's contents},
label=lst:asynchronous_call]{lst/asynchronous_call.txt}

This concept is coupled with an event loop, which is a single thread that is
running inside the main process.
The loop constantly checks for new events. When an event is detected, the loop
invokes the corresponding callback function. The callback is processed in the
same thread which means that there is at most one callback running at a time.
The event loop continues when the callback has completed. As a result the
developer does not need to take care of concurrency issues during development.
But the developer's task is to write light event handlers that can be processed
quickly as every callback is an interruption of the event processing in the
event loop. \footcite[Cf.][]{Croucher_2010} Memory or processor intense callbacks
can lead to growing queues of unserved events which eventuelly results
in a slow application or service\footcite[Cf.][48]{teixeira_2012}.


% implication: 
% all code should be as non-blocking and asynchronous as possible
% using blocking APIs inside the asynchronous code can cause blocking of the event loop
% which is evil. This is why node.js is based on javacript. Other languages already have
% lots of modules with blocking APIs which could confuse developers.



\subsection{Existing Asynchronous Frameworks}
\label{existing_frameworks}


\subsubsection{Overview}
\label{frameworks_overview}

\begin{savenotes} %alloows proper citation marks inside tables
\begin{table}[h]
\centering
\begin{tabular*}{\textwidth}{p{0.2\textwidth} p{0.15\textwidth} p{0.57\textwidth}}
\toprule
\textbf{Name} & \textbf{Language/s} & \textbf{Description} \\
\midrule 
Twisted			& Python			& ``Twisted is an event-driven networking engine written in 
									  Python and licensed under the open source MIT license"\footcite[Cf.][]{Twisted_2012}.
									  Twisted is not fully threadsafe and implements the event
									  loop via a library.\\
									  
EventMachine	& Ruby				& Similar to Twisted.\\

Node.js			& JavaScript		& Node is based on Chrome's JavaScript runtime \textit{V8}.
									  The platform is fully event-driven and offers core
									  functionalities for network programming. Its functionality
									  can be extended with a large number of modules using an
									  integrated package management system.
									  Node.js started development in 2009 and the community
									  is growing since that.\footcite[Cf.][]{Mashtable_2011}\\
									  
Vert.x			& JavaScript, Java, Python, Groovy, Ruby, Coffeescript		
									& A JVM based platform for network applications which is
									  inspired by Node.js.\\
\bottomrule 
\end{tabular*}
  \captionof{table}{Existing asynchronous programming frameworks}
  \label{tab:existing_frameworks}
\end{table}
\end{savenotes}



\subsubsection{Node.js}
\label{node.js}



\subsubsection{Vert.x}
\label{vert.x}




\section{Areas of Application}
\label{areas_of_application}

The non-blocking nature of asynchronous calls is important in all types of
applications that need to handle a large number of requests in real time.

Some success stories are available on \url{http://nodejs.org/}.


\section{Exemplary Implementations}
\label{exemplary_implementations}

A simple web form application has been implemented in Node.js and Vert.x to
further analyze non-functional requirements and collect practical experience
with these frameworks.

\subsection{Software Description}
\label{software_description}
Brief description of the insurance fee calculator

\subsection{Software Design}
\label{software_design}
High level design\\
Interface description, and differences between Node.js and Vert.x

\subsection{Software Implementation}
\label{software_implementation}
Complications or any other notes on the implementation process that might be of
importance for the evaluation.





\section{Evaluation of Non-functional Attributes}
\label{evaluation_nonfunctional}

\subsection{Maintainability}
\label{maintainability}
Language: JavaScript is wide spread, same for Java. However JavaScript offers
better flexibility and native constructs for these types of applications.
(Java 8 might improve things a bit).
Node's API is undergoing backwards incompatible changes from time to time.
It is desired to updated some components of a Node based application from
time to time.

\subsection{Integration}
\label{integration}
Vert.x and Node.js are supposed to be used as fully event driven standalone
applications that can be extended with event-driven modules.
However it might be desired to reuse existing software that is not fully event
driven.
Dont make use of blocking apis.
Consider connecting that software with a message bus.
Separate IO intense tasks into ``web workers'' or similar constructs.

\subsection{Scalability}
\label{scalability}
Support for single machine scaling using multiple threads exists.
Deployment on distributed systems differs in node.js and vert.x.
Vert.x. uses its own communication bus to share information between verticles.


\section{Conclusion}
\label{conclusion}
