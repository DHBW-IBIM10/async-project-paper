

\title{\papertitle}
\author{Rocco Schulz, Max Vökler, Joe Boden,\\ Robert Wawrzyniak, Can Paul Bineytioglu\\\\
	Corporate State University\\Baden-Wuerttemberg - Stuttgart}

\date{\today}

\begin{document}

\maketitle


\begin{abstract}
This paper evaluates asynchronous server technologies. Strenghts and weaknesses
of asynchronous programming models are elaborated and a proof of concept 
based on node.js and vert.x is used to evaluate non-functional attributes such as
maintainability. \ldots
\end{abstract}

\section{Introduction}
In traditional Web applications a connection's lifespan starts with a request and
ends with the servers response. Each connection is assigned a thread which
processes the request and terminates afterwards.
However in order to allow real-time\footnote{proper real time, as opposed to
ancient definitions. TODO: reference} communication between client and server it
is necessary to extend the connection's lifespan to the duration of the whole
session which results in many simultaniously open connections.
Assigning one thread for each open connection would cause a large overhead for
the server even though most threads would be inactive. This is due to the data
associated with each thread and the process of switching between all
threads.\footnote{TODO: there's probably a better term than switching}\\
Relatively young frameworks such as Node.js and Vert.x try to address this issue 
by providing a completely asynchronous programming model which allows associating
multiple simultanious connections with a single thread by using an event-driven approach.

%Goal of this paper and structure
This paper elaborates the concepts behind these young frameworks and analyses their 
technical strengths and weaknesses. Furthermore non-functional attributes will be
evaluated based on two sample implementations in Node.js  and Vert.x.

\section{Setting the context}
\label{setting_the_context}

\subsection{Comparison between Asynchronous and Synchronous Processing}
\label{comparison}
TODO: Benefits of this approach vs synchronous (literature: distributed systems)

\subsection{Existing Asynchronous Frameworks}
\label{existing_frameworks}
list of recent asynchrounous programming models and frameworks such as
vert.x, node.js, twisted, EventMachine, etc.

A table with short descriptions should be sufficient.



\section{Areas of Application}
\label{areas_of_application}




\section{Exemplary Implementations}
\label{exemplary_implementations}

A simple web form application has been implemented in Node.js and Vert.x to
further analyze non-functional requirements and collect practical experience
with these frameworks.

\subsection{Software Description}
\label{program_description}
Brief description of the insurance fee calculator

\subsection{Software Design}
\label{software_design}
High level design\\
Interface description, and differences between Node.js and Vert.x

\subsection{Software Implementation}
\label{program_description}
Complications or any other notes on the implementation process that might be
of importance for the evaluation.





\section{Evaluation of Non-functional Attributes}
\label{evaluation_nonfunctional}





\section{Conclusion}
\label{conclusion}




